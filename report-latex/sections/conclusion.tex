\section{Conclusion}
\label{sec:conclusion}

This paper presents a comparative evaluation of InterFuser and TransFuser, two state-of-the-art AI-based autonomous driving agents, within the CARLA simulation environment. The primary objective was to assess and contrast their performance and behavioral characteristics in diverse urban driving scenarios. Leveraging the standardized CARLA Leaderboard metrics, the study systematically analyzed aspects such as route completion, infraction rates, and driving safety. Additionally, qualitative video footage of the agents' driving sessions was reviewed to identify and characterize unusual or problematic behaviors not captured by quantitative metrics.

The evaluation revealed several common failure modes across both models, including premature stopping at intersections or traffic lights, failure to come to a complete stop at stop signs, becoming indefinitely blocked in the presence of nearby static or dynamic obstacles, severe degradation of performance under low-light conditions, unreliable or unsafe creeping behavior leading to collisions, and inadequate throttle control on upward slopes. The findings contribute to the broader effort of benchmarking autonomous driving policies in simulation and support the informed development of future sensor fusion architectures.